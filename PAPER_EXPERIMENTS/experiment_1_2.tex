% Experiment 1.2: Theoretical Predictions Validation
% Include this file with: % Experiment 1.2: Theoretical Predictions Validation
% Include this file with: % Experiment 1.2: Theoretical Predictions Validation
% Include this file with: % Experiment 1.2: Theoretical Predictions Validation
% Include this file with: \input{experiment_1_2.tex}

\subsection{Experiment 1.2: Theoretical Predictions on Real Networks}
\label{sec:exp1_2}

Having validated DSpar's behavior on synthetic networks with planted community structure (Section~\ref{sec:exp1_1}), we now examine whether the theoretical predictions hold on real-world networks where community structure emerges organically.

\paragraph{Experimental Setup.}
We evaluate DSpar on \num{12} real-world networks from the SNAP repository~\cite{snapnets}, spanning collaboration networks (ca-AstroPh, ca-CondMat, ca-GrQc, ca-HepPh, ca-HepTh), citation networks (cit-HepPh, cit-HepTh), email communication (email-Enron, email-Eu-core), social networks (facebook-combined, ego-Facebook), and voting networks (wiki-Vote).
For each network, we apply DSpar sparsification at retention levels $\alpha \in \{0.5, 0.6, 0.7, 0.8, 0.9\}$, with 10 independent trials per retention level.

We measure modularity change under two evaluation protocols:
\begin{enumerate}
    \item \textbf{Fixed-membership} ($\Delta Q_{\mathrm{fixed}}$): Compute modularity on the sparsified graph using the community assignment from the original graph. This isolates the mechanism predicted by theory.
    \item \textbf{Leiden-reoptimized} ($\Delta Q_{\mathrm{Leiden}}$): Apply Leiden community detection~\cite{traag2019leiden} to the sparsified graph and compute modularity with the new partition. This measures downstream pipeline performance.
\end{enumerate}

\paragraph{Theoretical Identity.}
Modularity decomposes as $Q = F - G$, where $F = \frac{1}{2m}\sum_c e_c$ captures intra-community edge density and $G = \frac{1}{4m^2}\sum_c \mathrm{vol}_c^2$ is the null-model penalty.
The change under sparsification satisfies:
\begin{equation}
    \Delta Q_{\mathrm{fixed}} = \Delta F_{\mathrm{obs}} - \Delta G_{\mathrm{obs}}
    \label{eq:decomposition}
\end{equation}
where $\Delta F_{\mathrm{obs}}$ reflects changes in intra-community edge fraction and $\Delta G_{\mathrm{obs}}$ reflects changes in the null-model penalty term.

% === RESULTS ===

\paragraph{Results.}
Table~\ref{tab:exp1_2_modularity} summarizes the modularity changes at $\alpha = 0.8$ across all datasets.
% TODO: Summarize key findings from the table here.
% Example: "DSpar consistently improves fixed-membership modularity across X/12 datasets..."

\input{results/exp1_2_theoretical/tables/table1_modularity_changes.tex}

% === DECOMPOSITION VERIFICATION ===

\paragraph{Decomposition Verification.}
Table~\ref{tab:exp1_2_decomposition} verifies the identity in Equation~\eqref{eq:decomposition}.
The reconstruction error $|\epsilon| = |\Delta Q_{\mathrm{fixed}} - (\Delta F_{\mathrm{obs}} - \Delta G_{\mathrm{obs}})|$ is at machine precision for all datasets, confirming our theoretical analysis.

\input{results/exp1_2_theoretical/tables/table2_decomposition.tex}

% === FIGURES ===

\paragraph{Modularity vs.\ Retention.}
Figure~\ref{fig:exp1_2_modularity} shows the relationship between retention $\alpha$ and modularity change for representative datasets.
% TODO: Select 2-4 representative datasets for the main figure.
% TODO: Move remaining datasets to supplementary material if needed.

\begin{figure}[htbp]
    \centering
    % TODO: Replace with actual figure paths from plot_exp1_2.py output
    % \includegraphics[width=0.48\textwidth]{results/exp1_2_theoretical/figures/ca-AstroPh_figure_A_modularity.pdf}
    % \includegraphics[width=0.48\textwidth]{results/exp1_2_theoretical/figures/email-Enron_figure_A_modularity.pdf}
    \caption{
        Modularity change vs.\ retention for representative datasets.
        \textbf{Blue circles}: $\Delta Q_{\mathrm{fixed}}$ (theory-aligned, fixed membership).
        \textbf{Orange squares}: $\Delta Q_{\mathrm{Leiden}}$ (downstream pipeline performance).
        Error bars indicate standard deviation over 10 trials.
        % TODO: Add interpretation of the trend.
    }
    \label{fig:exp1_2_modularity}
\end{figure}

\paragraph{Modularity Decomposition.}
Figure~\ref{fig:exp1_2_decomposition} illustrates the decomposition $\Delta Q = \Delta F - \Delta G$ for selected datasets.
The positive contribution $-\Delta G$ (null-model relief) demonstrates that DSpar preferentially removes edges that inflate the degree-based penalty term.

\begin{figure}[htbp]
    \centering
    % TODO: Replace with actual figure paths
    % \includegraphics[width=0.48\textwidth]{results/exp1_2_theoretical/figures/ca-AstroPh_figure_B_decomposition.pdf}
    % \includegraphics[width=0.48\textwidth]{results/exp1_2_theoretical/figures/email-Enron_figure_B_decomposition.pdf}
    \caption{
        Modularity decomposition for representative datasets.
        \textbf{Blue circles}: $\Delta Q_{\mathrm{fixed}}$ (total change).
        \textbf{Green triangles}: $-\Delta G$ (null-model relief, shown as positive contribution).
        The identity $\Delta Q = \Delta F - \Delta G$ holds to machine precision.
    }
    \label{fig:exp1_2_decomposition}
\end{figure}

% === SUMMARY STATISTICS ===

\paragraph{Summary.}
Table~\ref{tab:exp1_2_summary} provides aggregate statistics across all datasets.
% TODO: Discuss the overall success rate and any outliers.

\input{results/exp1_2_theoretical/tables/table3_summary.tex}

% === DISCUSSION ===

\paragraph{Discussion.}
% TODO: Add discussion points:
% 1. Compare fixed-membership vs Leiden results
% 2. Explain any datasets where modularity decreased
% 3. Relate findings back to the theoretical predictions
% 4. Discuss practical implications for community detection pipelines

% === OPTIONAL: COMBINED FIGURE FOR SUPPLEMENTARY ===

% If using the combined multi-panel figure:
% \begin{figure*}[htbp]
%     \centering
%     \includegraphics[width=\textwidth]{results/exp1_2_theoretical/figures/all_datasets_modularity_combined.pdf}
%     \caption{
%         Modularity change vs.\ retention for all 12 datasets.
%         Blue: $\Delta Q_{\mathrm{fixed}}$, Orange: $\Delta Q_{\mathrm{Leiden}}$.
%     }
%     \label{fig:exp1_2_all}
% \end{figure*}


\subsection{Experiment 1.2: Theoretical Predictions on Real Networks}
\label{sec:exp1_2}

Having validated DSpar's behavior on synthetic networks with planted community structure (Section~\ref{sec:exp1_1}), we now examine whether the theoretical predictions hold on real-world networks where community structure emerges organically.

\paragraph{Experimental Setup.}
We evaluate DSpar on \num{12} real-world networks from the SNAP repository~\cite{snapnets}, spanning collaboration networks (ca-AstroPh, ca-CondMat, ca-GrQc, ca-HepPh, ca-HepTh), citation networks (cit-HepPh, cit-HepTh), email communication (email-Enron, email-Eu-core), social networks (facebook-combined, ego-Facebook), and voting networks (wiki-Vote).
For each network, we apply DSpar sparsification at retention levels $\alpha \in \{0.5, 0.6, 0.7, 0.8, 0.9\}$, with 10 independent trials per retention level.

We measure modularity change under two evaluation protocols:
\begin{enumerate}
    \item \textbf{Fixed-membership} ($\Delta Q_{\mathrm{fixed}}$): Compute modularity on the sparsified graph using the community assignment from the original graph. This isolates the mechanism predicted by theory.
    \item \textbf{Leiden-reoptimized} ($\Delta Q_{\mathrm{Leiden}}$): Apply Leiden community detection~\cite{traag2019leiden} to the sparsified graph and compute modularity with the new partition. This measures downstream pipeline performance.
\end{enumerate}

\paragraph{Theoretical Identity.}
Modularity decomposes as $Q = F - G$, where $F = \frac{1}{2m}\sum_c e_c$ captures intra-community edge density and $G = \frac{1}{4m^2}\sum_c \mathrm{vol}_c^2$ is the null-model penalty.
The change under sparsification satisfies:
\begin{equation}
    \Delta Q_{\mathrm{fixed}} = \Delta F_{\mathrm{obs}} - \Delta G_{\mathrm{obs}}
    \label{eq:decomposition}
\end{equation}
where $\Delta F_{\mathrm{obs}}$ reflects changes in intra-community edge fraction and $\Delta G_{\mathrm{obs}}$ reflects changes in the null-model penalty term.

% === RESULTS ===

\paragraph{Results.}
Table~\ref{tab:exp1_2_modularity} summarizes the modularity changes at $\alpha = 0.8$ across all datasets.
% TODO: Summarize key findings from the table here.
% Example: "DSpar consistently improves fixed-membership modularity across X/12 datasets..."

\input{results/exp1_2_theoretical/tables/table1_modularity_changes.tex}

% === DECOMPOSITION VERIFICATION ===

\paragraph{Decomposition Verification.}
Table~\ref{tab:exp1_2_decomposition} verifies the identity in Equation~\eqref{eq:decomposition}.
The reconstruction error $|\epsilon| = |\Delta Q_{\mathrm{fixed}} - (\Delta F_{\mathrm{obs}} - \Delta G_{\mathrm{obs}})|$ is at machine precision for all datasets, confirming our theoretical analysis.

\input{results/exp1_2_theoretical/tables/table2_decomposition.tex}

% === FIGURES ===

\paragraph{Modularity vs.\ Retention.}
Figure~\ref{fig:exp1_2_modularity} shows the relationship between retention $\alpha$ and modularity change for representative datasets.
% TODO: Select 2-4 representative datasets for the main figure.
% TODO: Move remaining datasets to supplementary material if needed.

\begin{figure}[htbp]
    \centering
    % TODO: Replace with actual figure paths from plot_exp1_2.py output
    % \includegraphics[width=0.48\textwidth]{results/exp1_2_theoretical/figures/ca-AstroPh_figure_A_modularity.pdf}
    % \includegraphics[width=0.48\textwidth]{results/exp1_2_theoretical/figures/email-Enron_figure_A_modularity.pdf}
    \caption{
        Modularity change vs.\ retention for representative datasets.
        \textbf{Blue circles}: $\Delta Q_{\mathrm{fixed}}$ (theory-aligned, fixed membership).
        \textbf{Orange squares}: $\Delta Q_{\mathrm{Leiden}}$ (downstream pipeline performance).
        Error bars indicate standard deviation over 10 trials.
        % TODO: Add interpretation of the trend.
    }
    \label{fig:exp1_2_modularity}
\end{figure}

\paragraph{Modularity Decomposition.}
Figure~\ref{fig:exp1_2_decomposition} illustrates the decomposition $\Delta Q = \Delta F - \Delta G$ for selected datasets.
The positive contribution $-\Delta G$ (null-model relief) demonstrates that DSpar preferentially removes edges that inflate the degree-based penalty term.

\begin{figure}[htbp]
    \centering
    % TODO: Replace with actual figure paths
    % \includegraphics[width=0.48\textwidth]{results/exp1_2_theoretical/figures/ca-AstroPh_figure_B_decomposition.pdf}
    % \includegraphics[width=0.48\textwidth]{results/exp1_2_theoretical/figures/email-Enron_figure_B_decomposition.pdf}
    \caption{
        Modularity decomposition for representative datasets.
        \textbf{Blue circles}: $\Delta Q_{\mathrm{fixed}}$ (total change).
        \textbf{Green triangles}: $-\Delta G$ (null-model relief, shown as positive contribution).
        The identity $\Delta Q = \Delta F - \Delta G$ holds to machine precision.
    }
    \label{fig:exp1_2_decomposition}
\end{figure}

% === SUMMARY STATISTICS ===

\paragraph{Summary.}
Table~\ref{tab:exp1_2_summary} provides aggregate statistics across all datasets.
% TODO: Discuss the overall success rate and any outliers.

\input{results/exp1_2_theoretical/tables/table3_summary.tex}

% === DISCUSSION ===

\paragraph{Discussion.}
% TODO: Add discussion points:
% 1. Compare fixed-membership vs Leiden results
% 2. Explain any datasets where modularity decreased
% 3. Relate findings back to the theoretical predictions
% 4. Discuss practical implications for community detection pipelines

% === OPTIONAL: COMBINED FIGURE FOR SUPPLEMENTARY ===

% If using the combined multi-panel figure:
% \begin{figure*}[htbp]
%     \centering
%     \includegraphics[width=\textwidth]{results/exp1_2_theoretical/figures/all_datasets_modularity_combined.pdf}
%     \caption{
%         Modularity change vs.\ retention for all 12 datasets.
%         Blue: $\Delta Q_{\mathrm{fixed}}$, Orange: $\Delta Q_{\mathrm{Leiden}}$.
%     }
%     \label{fig:exp1_2_all}
% \end{figure*}


\subsection{Experiment 1.2: Theoretical Predictions on Real Networks}
\label{sec:exp1_2}

Having validated DSpar's behavior on synthetic networks with planted community structure (Section~\ref{sec:exp1_1}), we now examine whether the theoretical predictions hold on real-world networks where community structure emerges organically.

\paragraph{Experimental Setup.}
We evaluate DSpar on \num{12} real-world networks from the SNAP repository~\cite{snapnets}, spanning collaboration networks (ca-AstroPh, ca-CondMat, ca-GrQc, ca-HepPh, ca-HepTh), citation networks (cit-HepPh, cit-HepTh), email communication (email-Enron, email-Eu-core), social networks (facebook-combined, ego-Facebook), and voting networks (wiki-Vote).
For each network, we apply DSpar sparsification at retention levels $\alpha \in \{0.5, 0.6, 0.7, 0.8, 0.9\}$, with 10 independent trials per retention level.

We measure modularity change under two evaluation protocols:
\begin{enumerate}
    \item \textbf{Fixed-membership} ($\Delta Q_{\mathrm{fixed}}$): Compute modularity on the sparsified graph using the community assignment from the original graph. This isolates the mechanism predicted by theory.
    \item \textbf{Leiden-reoptimized} ($\Delta Q_{\mathrm{Leiden}}$): Apply Leiden community detection~\cite{traag2019leiden} to the sparsified graph and compute modularity with the new partition. This measures downstream pipeline performance.
\end{enumerate}

\paragraph{Theoretical Identity.}
Modularity decomposes as $Q = F - G$, where $F = \frac{1}{2m}\sum_c e_c$ captures intra-community edge density and $G = \frac{1}{4m^2}\sum_c \mathrm{vol}_c^2$ is the null-model penalty.
The change under sparsification satisfies:
\begin{equation}
    \Delta Q_{\mathrm{fixed}} = \Delta F_{\mathrm{obs}} - \Delta G_{\mathrm{obs}}
    \label{eq:decomposition}
\end{equation}
where $\Delta F_{\mathrm{obs}}$ reflects changes in intra-community edge fraction and $\Delta G_{\mathrm{obs}}$ reflects changes in the null-model penalty term.

% === RESULTS ===

\paragraph{Results.}
Table~\ref{tab:exp1_2_modularity} summarizes the modularity changes at $\alpha = 0.8$ across all datasets.
% TODO: Summarize key findings from the table here.
% Example: "DSpar consistently improves fixed-membership modularity across X/12 datasets..."

\input{results/exp1_2_theoretical/tables/table1_modularity_changes.tex}

% === DECOMPOSITION VERIFICATION ===

\paragraph{Decomposition Verification.}
Table~\ref{tab:exp1_2_decomposition} verifies the identity in Equation~\eqref{eq:decomposition}.
The reconstruction error $|\epsilon| = |\Delta Q_{\mathrm{fixed}} - (\Delta F_{\mathrm{obs}} - \Delta G_{\mathrm{obs}})|$ is at machine precision for all datasets, confirming our theoretical analysis.

\input{results/exp1_2_theoretical/tables/table2_decomposition.tex}

% === FIGURES ===

\paragraph{Modularity vs.\ Retention.}
Figure~\ref{fig:exp1_2_modularity} shows the relationship between retention $\alpha$ and modularity change for representative datasets.
% TODO: Select 2-4 representative datasets for the main figure.
% TODO: Move remaining datasets to supplementary material if needed.

\begin{figure}[htbp]
    \centering
    % TODO: Replace with actual figure paths from plot_exp1_2.py output
    % \includegraphics[width=0.48\textwidth]{results/exp1_2_theoretical/figures/ca-AstroPh_figure_A_modularity.pdf}
    % \includegraphics[width=0.48\textwidth]{results/exp1_2_theoretical/figures/email-Enron_figure_A_modularity.pdf}
    \caption{
        Modularity change vs.\ retention for representative datasets.
        \textbf{Blue circles}: $\Delta Q_{\mathrm{fixed}}$ (theory-aligned, fixed membership).
        \textbf{Orange squares}: $\Delta Q_{\mathrm{Leiden}}$ (downstream pipeline performance).
        Error bars indicate standard deviation over 10 trials.
        % TODO: Add interpretation of the trend.
    }
    \label{fig:exp1_2_modularity}
\end{figure}

\paragraph{Modularity Decomposition.}
Figure~\ref{fig:exp1_2_decomposition} illustrates the decomposition $\Delta Q = \Delta F - \Delta G$ for selected datasets.
The positive contribution $-\Delta G$ (null-model relief) demonstrates that DSpar preferentially removes edges that inflate the degree-based penalty term.

\begin{figure}[htbp]
    \centering
    % TODO: Replace with actual figure paths
    % \includegraphics[width=0.48\textwidth]{results/exp1_2_theoretical/figures/ca-AstroPh_figure_B_decomposition.pdf}
    % \includegraphics[width=0.48\textwidth]{results/exp1_2_theoretical/figures/email-Enron_figure_B_decomposition.pdf}
    \caption{
        Modularity decomposition for representative datasets.
        \textbf{Blue circles}: $\Delta Q_{\mathrm{fixed}}$ (total change).
        \textbf{Green triangles}: $-\Delta G$ (null-model relief, shown as positive contribution).
        The identity $\Delta Q = \Delta F - \Delta G$ holds to machine precision.
    }
    \label{fig:exp1_2_decomposition}
\end{figure}

% === SUMMARY STATISTICS ===

\paragraph{Summary.}
Table~\ref{tab:exp1_2_summary} provides aggregate statistics across all datasets.
% TODO: Discuss the overall success rate and any outliers.

\input{results/exp1_2_theoretical/tables/table3_summary.tex}

% === DISCUSSION ===

\paragraph{Discussion.}
% TODO: Add discussion points:
% 1. Compare fixed-membership vs Leiden results
% 2. Explain any datasets where modularity decreased
% 3. Relate findings back to the theoretical predictions
% 4. Discuss practical implications for community detection pipelines

% === OPTIONAL: COMBINED FIGURE FOR SUPPLEMENTARY ===

% If using the combined multi-panel figure:
% \begin{figure*}[htbp]
%     \centering
%     \includegraphics[width=\textwidth]{results/exp1_2_theoretical/figures/all_datasets_modularity_combined.pdf}
%     \caption{
%         Modularity change vs.\ retention for all 12 datasets.
%         Blue: $\Delta Q_{\mathrm{fixed}}$, Orange: $\Delta Q_{\mathrm{Leiden}}$.
%     }
%     \label{fig:exp1_2_all}
% \end{figure*}


\subsection{Experiment 1.2: Theoretical Predictions on Real Networks}
\label{sec:exp1_2}

Having validated DSpar's behavior on synthetic networks with planted community structure (Section~\ref{sec:exp1_1}), we now examine whether the theoretical predictions hold on real-world networks where community structure emerges organically.

\paragraph{Experimental Setup.}
We evaluate DSpar on \num{12} real-world networks from the SNAP repository~\cite{snapnets}, spanning collaboration networks (ca-AstroPh, ca-CondMat, ca-GrQc, ca-HepPh, ca-HepTh), citation networks (cit-HepPh, cit-HepTh), email communication (email-Enron, email-Eu-core), social networks (facebook-combined, ego-Facebook), and voting networks (wiki-Vote).
For each network, we apply DSpar sparsification at retention levels $\alpha \in \{0.5, 0.6, 0.7, 0.8, 0.9\}$, with 10 independent trials per retention level.

We measure modularity change under two evaluation protocols:
\begin{enumerate}
    \item \textbf{Fixed-membership} ($\Delta Q_{\mathrm{fixed}}$): Compute modularity on the sparsified graph using the community assignment from the original graph. This isolates the mechanism predicted by theory.
    \item \textbf{Leiden-reoptimized} ($\Delta Q_{\mathrm{Leiden}}$): Apply Leiden community detection~\cite{traag2019leiden} to the sparsified graph and compute modularity with the new partition. This measures downstream pipeline performance.
\end{enumerate}

\paragraph{Theoretical Identity.}
Modularity decomposes as $Q = F - G$, where $F = \frac{1}{2m}\sum_c e_c$ captures intra-community edge density and $G = \frac{1}{4m^2}\sum_c \mathrm{vol}_c^2$ is the null-model penalty.
The change under sparsification satisfies:
\begin{equation}
    \Delta Q_{\mathrm{fixed}} = \Delta F_{\mathrm{obs}} - \Delta G_{\mathrm{obs}}
    \label{eq:decomposition}
\end{equation}
where $\Delta F_{\mathrm{obs}}$ reflects changes in intra-community edge fraction and $\Delta G_{\mathrm{obs}}$ reflects changes in the null-model penalty term.

% === RESULTS ===

\paragraph{Results.}
Table~\ref{tab:exp1_2_modularity} summarizes the modularity changes at $\alpha = 0.8$ across all datasets.
% TODO: Summarize key findings from the table here.
% Example: "DSpar consistently improves fixed-membership modularity across X/12 datasets..."

\input{results/exp1_2_theoretical/tables/table1_modularity_changes.tex}

% === DECOMPOSITION VERIFICATION ===

\paragraph{Decomposition Verification.}
Table~\ref{tab:exp1_2_decomposition} verifies the identity in Equation~\eqref{eq:decomposition}.
The reconstruction error $|\epsilon| = |\Delta Q_{\mathrm{fixed}} - (\Delta F_{\mathrm{obs}} - \Delta G_{\mathrm{obs}})|$ is at machine precision for all datasets, confirming our theoretical analysis.

\input{results/exp1_2_theoretical/tables/table2_decomposition.tex}

% === FIGURES ===

\paragraph{Modularity vs.\ Retention.}
Figure~\ref{fig:exp1_2_modularity} shows the relationship between retention $\alpha$ and modularity change for representative datasets.
% TODO: Select 2-4 representative datasets for the main figure.
% TODO: Move remaining datasets to supplementary material if needed.

\begin{figure}[htbp]
    \centering
    % TODO: Replace with actual figure paths from plot_exp1_2.py output
    % \includegraphics[width=0.48\textwidth]{results/exp1_2_theoretical/figures/ca-AstroPh_figure_A_modularity.pdf}
    % \includegraphics[width=0.48\textwidth]{results/exp1_2_theoretical/figures/email-Enron_figure_A_modularity.pdf}
    \caption{
        Modularity change vs.\ retention for representative datasets.
        \textbf{Blue circles}: $\Delta Q_{\mathrm{fixed}}$ (theory-aligned, fixed membership).
        \textbf{Orange squares}: $\Delta Q_{\mathrm{Leiden}}$ (downstream pipeline performance).
        Error bars indicate standard deviation over 10 trials.
        % TODO: Add interpretation of the trend.
    }
    \label{fig:exp1_2_modularity}
\end{figure}

\paragraph{Modularity Decomposition.}
Figure~\ref{fig:exp1_2_decomposition} illustrates the decomposition $\Delta Q = \Delta F - \Delta G$ for selected datasets.
The positive contribution $-\Delta G$ (null-model relief) demonstrates that DSpar preferentially removes edges that inflate the degree-based penalty term.

\begin{figure}[htbp]
    \centering
    % TODO: Replace with actual figure paths
    % \includegraphics[width=0.48\textwidth]{results/exp1_2_theoretical/figures/ca-AstroPh_figure_B_decomposition.pdf}
    % \includegraphics[width=0.48\textwidth]{results/exp1_2_theoretical/figures/email-Enron_figure_B_decomposition.pdf}
    \caption{
        Modularity decomposition for representative datasets.
        \textbf{Blue circles}: $\Delta Q_{\mathrm{fixed}}$ (total change).
        \textbf{Green triangles}: $-\Delta G$ (null-model relief, shown as positive contribution).
        The identity $\Delta Q = \Delta F - \Delta G$ holds to machine precision.
    }
    \label{fig:exp1_2_decomposition}
\end{figure}

% === SUMMARY STATISTICS ===

\paragraph{Summary.}
Table~\ref{tab:exp1_2_summary} provides aggregate statistics across all datasets.
% TODO: Discuss the overall success rate and any outliers.

\input{results/exp1_2_theoretical/tables/table3_summary.tex}

% === DISCUSSION ===

\paragraph{Discussion.}
% TODO: Add discussion points:
% 1. Compare fixed-membership vs Leiden results
% 2. Explain any datasets where modularity decreased
% 3. Relate findings back to the theoretical predictions
% 4. Discuss practical implications for community detection pipelines

% === OPTIONAL: COMBINED FIGURE FOR SUPPLEMENTARY ===

% If using the combined multi-panel figure:
% \begin{figure*}[htbp]
%     \centering
%     \includegraphics[width=\textwidth]{results/exp1_2_theoretical/figures/all_datasets_modularity_combined.pdf}
%     \caption{
%         Modularity change vs.\ retention for all 12 datasets.
%         Blue: $\Delta Q_{\mathrm{fixed}}$, Orange: $\Delta Q_{\mathrm{Leiden}}$.
%     }
%     \label{fig:exp1_2_all}
% \end{figure*}
